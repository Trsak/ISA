\documentclass[a4paper, 11pt]{article}
\usepackage[czech]{babel}
\usepackage[utf8]{inputenc}
\usepackage[text={17cm, 24cm}, lmargin=2cm, tmargin=3cm]{geometry}
\bibliographystyle{czplain}
\usepackage{url}
\DeclareUrlCommand\url{\def\UrlLeft{<}\def\UrlRight{>} \urlstyle{tt}}

\begin{document}
    \begin{titlepage}
        \begin{center}
            \textsc{\Huge Vysoké učení technické v~Brně}\\[0.6em]
            \textsc{\huge Fakulta informačních technologií}\\
            \vspace{\stretch{0.382}}
            {\LARGE Dokumentace k projektu} \\[0.3em] {\Huge Export DNS informací pomocí protokolu Syslog}\\
            \vspace{\stretch{0.618}}
        \end{center}
        {\Large \today \hfill
        Petr Šopf (xsopfp00)}
    \end{titlepage}
    
 
\newpage
  \tableofcontents
\newpage

\section{Historie programování}
Nejstarším známým kalkulátorem je mechanismus z~Antikythéry pocházející ze starověkého Řecka. Vyroben byl přibližně v~roce 150-100 př. n. l. a uměl předpovídat zatmění Měsíce a Slunce \cite{Solla:AnAncientGreekComputer}. Al-Džazárí v~roce 1206 sepsal knihu \emph{Kniha znalostí důmyslných mechanických zařízení}, ve které popsal programovatelný automat bubeníka \cite{Dzazari:KnihaZnalostiDumyslnychMechanickychZarizeni}. Nových rozměrů začalo programování nabývat v~roce 1801, kdy Joseph Marie Jacquard dokázal řídit svůj tkalcovský stav za pomocí děrných štítků \cite{Essinger:BirthOfTheInformationAge}. Kolem roku 1830 poté využil děrné štítky i Charles Babbage pro svůj analytický stroj \cite{ThoughtCo:TheFirstComputer}. První známou ženou zabývající se programováním byla Ada Lovelace \cite{Biography:AdaLovelaceBiography}.

Prvním vysokoúrovňovým programovacím jazykem byl Plankalkül, který vytvořil německý inženýr Konrad Zuse v~průběhu 2. světové války v~letech 1942 až 1945 \cite{Bauer:Plankalkul}. Tento jazyk byl určený pro potřeby nacistického Německa. Prvním komerčně dostupným programovacím jazykem byl jazyk FORTAN, vytvořen v~roce 1956 zaměstnancem IBM Johnem Backusem \cite{Wikipedie:Fortan}.

\section{Programování dnes}
Porovnání dnešních moderních programovacích jazyků a těch starších popsal ve své bakalářské práci z~roku 2010 Vavruška Pavel \cite{Thesis:Vavruska}. Vzhledem k~rychlému vývoji hardware je v~současné době kladen i velký důraz na paralelní programování \cite{ComputingInScienceEngineering:Concurrency}. Programování se rozšiřuje stále mezi více lidí a alespoň základy se vyučují na většině středních škol. Například programování malých robotů za pomocí robotických stavebnic nabízí i některé základní školy \cite{Thesis:Fiser}.
\newpage
\bibliography{citace}
\end{document}